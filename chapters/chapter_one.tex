%------------------------ Bab 1. Pendahuluan -----------------------
\newpage

\newcommand{\babone}{BAB 1. PENDAHULUAN}

\bc
\section*{\babone}
\addcontentsline{toc}{section}{\babone}
\ec

%------------------------ 1.1 Latar Belakang -----------------------
\setlength{\parindent}{6ex}
\subsection{Latar Belakang}

Just do a yappanese here, \lipsum

Contoh citing yaitu dengan cite atau autocite, seperti berikut outputnya \cite{pande_crop_2021}. wgwgwgwgwgwg, okay... all good.

%------------------------ 1.2 Rumusan Masalah -----------------------
\subsection{Rumusan Masalah}
\begin{enumerate}
  \item Yang pertama kira kira kenapa nih kok bisa gini ya? 
  \item Lebih spesifik apa ya yang mau diselesaikan?
  \item Lalu bagaimana ya untuk mewujudkan yang ingin diselesaikan?
\end{enumerate}

%------------------------ 1.3 Tujuan Penelitian -----------------------
\subsection{Tujuan Penelitian}
\begin{enumerate}
  \item Jadi tujuan agar gak gini itu ntar kita akan giniin.
  \item Nah solusi yang akan kita ingin selesaikan adalah dengan berikut.
  \item Untuk mewujudkan solusi tersebut kita akan begini dan begitu.
\end{enumerate}

%------------------------ 1.4 Manfaat Penelitian -----------------------
\subsection{Manfaat Penelitian}
\begin{enumerate}
  \item Memperluas peluang untuk menghasilkan varietas tanaman unggul yang nantinya akan meningkatkan keamanan pangan.
  \item Membantu dalam penghematan waktu dan biaya yang sebelumnya diperlukan untuk mengembangkan varietas benih baru.
  \item Sebagai referensi untuk penelitian selanjutnya.
\end{enumerate}

%------------------------ 1.5 Batasan Penelitian -----------------------
% TODO: disable karena tidak boleh ada batasan masalah untuk sempro
% \subsection{Batasan Penelitian}
% \begin{enumerate}
%   \item Dataset atau kumpulan data yang digunakan untuk melatih model ini berasal dari banyak sumber yang berbeda, terutama dari berbagai artikel dan jurnal ilmiah yang tersedia secara online. Sebagian dari sumber-sumber tersebut berasal dari penelitian dan kajian.
%   \item Untuk memprediksi hasil, hanya ada empat parameter utama yang digunakan yaitu spesies tanaman, durasi perendaman, dan konsentrasi EMS. Variabel tambahan yang dapat ditambahkan untuk konteks dan penyempurnaan. Misalnya, ada opsi untuk menunjukkan durasi pasca-perendaman, yang menunjukkan proses perendaman kedua untuk memfasilitasi efek mutagenik lebih lanjut. Namun, parameter tambahan ini tetap dianggap opsional. Sebagian besar dataset bergantung pada tiga parameter utama yang disebutkan sebelumnya untuk menjaga dasar analisis yang fokus dan konsisten.
%   \item Karena keterbatasan data yang tersedia dalam jurnal ilmiah, penelitian ilmiah, terutama dalam bidang biologi dan taksonomi, sering kali difokuskan pada sejumlah spesies tertentu yang menjadi objek studi yang penting atau relevan. Akibatnya, dukungan aplikasi terbatas pada beberapa spesies saja.
% \end{enumerate}
